\section{Implementation}
This project was implemented by using Apache Spark \cite{ApacheSpark}, a distributed computing framework which is great for processing large scale data. 
The implementation followed a systematic approach comprising several key stages of data preprocessing, model development, and analysis.

\subsection{Data Preprocessing}
The initial preprocessing stage involved critical steps to ensure data quality and consistency. 
I began by removing columns that i thought was unnecessary, additionally i removed cancelled flights to focus on actual delay patterns, as these represent anomalies rather than predictable delays. 
Missing values in key fields including arrival time, elapsed time, and delay related columns were filled by a 0.
I combined different types of delays into one "TotalDelay" feature to better predict and analyze flight delays in my model.

\subsection{Feature Selection and Engineering}
Based on correlation analysis and domain knowledge, these six key features were selected for the model:

\begin{itemize}
\item Departure Delay
\item Distance
\item Taxi-In Time
\item Taxi-Out Time
\item Carrier Delay
\item Weather Delay
\end{itemize}
These features were chosen after a analysis of their correlations with arrival delays, I found particularly strong relationships with taxi-out time (correlation: 0.322) and moderate relationships with taxi-in time (correlation: 0.123). Interestingly, distance showed only a weak positive correlation (0.007), which disrupt some of my assumptions about flight delays.

\subsection{Model Development}
The dataset was split using an 80-20 ratio, with 80\% used for training and 20\% for testing and evaluation. I implement a Linear Regression \cite{LR} model using PySpark's ML library \cite{SparkMLlib}, with features assembled into vectors using VectorAssembler. 
The model's hyperparameters was tuned using the CrossValidator with a 3-fold cross-validation approach.
The hyperparameter optimization explored different combinations of:

\begin{itemize}
\item RegParam: [0.01, 0.1, 1.0]
\item ElasticNetParam: [0.0, 0.5, 1.0]
\end{itemize}

The final model used the optimal parameters of RegParam = 0.01 and ElasticNetParam = 0.0, as determined by the Grid Search.

\subsection{Initial Data Analysis}
The initial analysis revealed several important patterns in flight delays:

\subsubsection{Time-based patterns}
Day-of-week analysis showed highest average delays on Day 5 (Friday) (10.95 minutes), while monthly analysis revealed December as having the highest delays (16.68 minutes). These patterns suggest significant time-based patterns in delay probabilities that could be valuable for both airlines and passengers in planning.

\subsubsection{Carrier Performance}
Significant variations were observed among carriers, with average delays ranging from -2.89 to 12.61 minutes. This variance indicates that carrier-specific factors play a substantial role in delay patterns, though these differences might also reflect variations in route networks and operational environments.
