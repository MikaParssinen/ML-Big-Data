\section{Discussion}
The results of the flight delay prediction model reveal several interesting patterns and insights about the nature of flight delays. 
The high $R^2$ value of 0.9739 suggests that the selected features effectively capture the majority of factors influencing flight delays. 
However, this strong performance must be considered alongside some notable limitations and observations.
The model's varying performance across different delay ranges is particularly noteworthy. 
Its higher accuracy for shorter delays (mean absolute error of 8-9 minutes) compared to longer delays (mean absolute error of 19.53 minutes for very high delays) suggests that extreme delays may involve complex, interconnected factors not fully captured by the current feature set.
This pattern aligns with industry observations that major delays often result from effects of multiple factors.
The feature importance analysis revealed some surprising findings. 
While the strong influence of departure delays and taxi times was expected, the minimal impact of distance on delays challenges common assumptions. 
This suggests that operational factors at airports, rather than flight duration, play a more crucial role in determining delays. 
Additionally, the relatively low importance of weather delays in the model might indicate that weather impacts are often captured indirectly through other features like departure delays and taxi times.