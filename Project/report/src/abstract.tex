\begin{abstract}
This study investigates flight delay prediction using machine learning techniques applied to the 2008 American Airlines dataset. 
By using Apache Spark for big data processing and linear regression modeling, I developed a predictive model achieving an $R^2$ value of 0.9739 and RMSE of 17.45 minutes.
The analysis revealed that departure delays, taxi times, and ground operations are the most significant predictors of total flight delays, while weather and carrier-specific factors showed lesser influence. 
The model demonstrated strong performance for shorter delays, with 50\% of predictions having errors under 8.39 minutes, though accuracy decreased for longer delays. Temporal analysis identified peak delay periods, with highest averages occurring on Day 5 and during December. These findings provide valuable insights for airlines and passengers, contributing to better delay prediction and management. The implementation methodology and results suggest potential for practical applications in flight scheduling and operations management, while also highlighting areas for future improvement in handling extreme delay cases and incorporating additional predictive factors.
\end{abstract}